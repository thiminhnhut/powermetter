\documentclass[13pt,border=12pt]{standalone}
\usepackage[utf8]{vietnam}
\usepackage{amsmath,amsfonts,amssymb}
\usepackage{flowchart}
\usetikzlibrary{arrows}

\begin{document}
	\begin{tikzpicture}[>=latex']
		\def\smbwd{6cm}
		
		\node (nguoncapvidieukhien) at (0, 0) [draw, process, align=center, minimum width=4cm, minimum height=1.5cm] {Nguồn AC/DC \\ \(220VAC/5VDC\)};

		\node (server) at (0, -2.5) [draw, process, align=center, minimum width=4cm, minimum height=1.5cm] {Server \\ Server Blynk};

		\node (app) at (0, -5) [draw, process, align=center, minimum width=4cm, minimum height=1.5cm] {App điều khiển trên điện thoại \\ App Blynk};

		\node (vidieukhien) at (6, 0) [draw, process, align=center, minimum width=4cm, minimum height=1.5cm] {Vi điều khiển \\ NodeMCU 1.0};

		\coordinate (point1) at (5, -0.75);

		\node (cambienpzem004t) at (12, 0) [draw, process, align=center, minimum width=4cm, minimum height=1.5cm] {Cảm biến dòng điện và điện áp \\ Module PZEM004T};

		\node (thietbidien) at (18, 0) [draw, process, align=center, minimum width=4cm, minimum height=1.5cm] {Thiết bị điện};

		\node (manhinhhienthi) at (12, -2.5) [draw, process, align=center, minimum width=4cm, minimum height=1.5cm] {Màn hình hiển thị \\ LCD 16x02};

		\node (relay) at (12, -5) [draw, process, align=center, minimum width=4cm, minimum height=1.5cm] {Điều khiển thiết bị \\ Module Relay 5VDC};

		\node (chuongbao) at (12, -7.5) [draw, process, align=center, minimum width=4cm, minimum height=1.5cm] {Cảnh báo quá tải \\ Chuông báo 5VDC};

		\node (nguoncongsuat) at (19, -7.5) [draw, process, align=center, minimum width=4cm, minimum height=1.5cm] {Nguồn 220vAC cấp cho thiết bị};
		
		\coordinate (point2) at (19, -0.75);

		\draw[->] (nguoncapvidieukhien) -- (vidieukhien);
		\draw[->] (vidieukhien) |- (manhinhhienthi);
		\draw[->] (vidieukhien) |- (relay);
		\draw[->] (relay) -| (thietbidien);
		\draw[->] (vidieukhien) |- (chuongbao);
		\draw[<->] (point1) |- (server);
		\draw[<->] (server) -- (app);
		\draw[<->] (vidieukhien) -- (cambienpzem004t);
		\draw[<-] (cambienpzem004t) -- (thietbidien);
		\draw[->] (nguoncongsuat) -- (point2); 
	\end{tikzpicture}
\end{document} 